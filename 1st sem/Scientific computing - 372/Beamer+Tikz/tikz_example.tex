\documentclass{article}
\usepackage{tikz}
\usetikzlibrary{calc}

\begin{document}

%\tikzstyle{c} = [draw,circle,minimum size=1cm]
%\tikzstyle{r} = [draw,rectangle,minimum size=.8cm]
%\tikzstyle{e} = [->,>=latex]
%\begin{tikzpicture}[scale=0.5]
%	\draw[help lines] (0,0) grid (10,10);
%	\node[c] (A) at (2,8) {A};
%	\node[c] (B) at (8,8) {B};
%	\node[c] (C) at (8,2) {C};
%	\node[r] (D) at (2,2) {D};
%	\draw (0,0) -- ($(A)!0.27!(C)$);
%	\draw[e] (A) -- (B);
%	\draw[e] (A) -- (C);
%	\draw[e] (B) -- (C);
%	\draw[e] (D) -- (C);
%\end{tikzpicture}

\tikzstyle{c} = [draw,circle,fill=blue,minimum size=1cm,node distance=3cm]
\tikzstyle{r} = [draw,rectangle,minimum size=.8cm,node distance=3cm]
\tikzstyle{e} = [->,>=latex]
\tikzstyle{l} = [above]
\begin{tikzpicture}[scale=0.5]
	\node[c] (A) at (100,100) {A};
	\node[c,right of=A] (B) {B};
	\node[c,below of=B] (C) {C};
	\node[r,left of=C] (D) {D};
	\draw[e] (A) to[bend left] node[l] {$\gamma$} (B);
	\draw[e] (A) -- (C);
	\draw[e] (B) -- (C);
	\draw[e] (D) -- node[l] {$\phi$} (C);
\end{tikzpicture}

\end{document}
